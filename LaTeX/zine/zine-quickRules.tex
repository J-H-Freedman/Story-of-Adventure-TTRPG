\documentclass[statementpaper,oneside,article,12pt]{memoir}
\usepackage{geometry}
\usepackage{libertine}
\usepackage{xcolor}
\usepackage{soul}

% Disable chapter/section numbering.
\setsecnumdepth{none}
\maxsecnumdepth{none}

% Optional background
% http://tex.stackexchange.com/a/276280
\usepackage{transparent}
\usepackage{eso-pic}
\newcommand{\BackgroundPic}[1]{%
	\put(0,0){%
		\parbox[b][\paperheight]{\paperwidth}{%
			\vfill
			\centering
			{\transparent{0.4} \includegraphics[width=\paperwidth,height=\paperheight,%
				keepaspectratio]{#1}}%
			\vfill
}}}

\begin{document}
	
	% Edit inside the { brackets } to change these.
	
	\title{\underline{Story of Adventure}:\\Quick Rules}
	\author{J.H. Freedman}
	\date{}
	
	\begingroup
	\let\cleardoublepage\clearpage
	
	% \AddToShipoutPicture*{\BackgroundPic{samplecover}}
	
	\begin{titlingpage}
		\maketitle
		
		A TTRPG Engine for Creativity\\Alpha 1.0
		
	\end{titlingpage}
	
	\endgroup
	
	% As the zine is so short, you probably won't need page numbers; however, if you
	% want them, comment out the next line with a %.
	\pagestyle{empty}
	
	
	%% CONTENT GOES BELOW
	
	\section*{Why would I play this?}
	%Do you and your friends want to create a serialized adventure with an ensemble cast of unique, bizarre characters? 
	Are you tired of supposed "classless" RPG basically being a one-class game of ticking universal, predetermined boxes?
	
	The core experience when playing Story of Adventure is creating outlandish characters with custom abilities who will be put in scenarios where the designated Story Teller (ST) throttles dramatic tension.
	
	Character creation is truly open with the freedom to create every ability from scratch and have them evolve with each level of power. Will you choose to develop several abilities, responding to the story's unfolding events? Or go all in on one signature ability and push it to its limit?
	
	\section*{How does it work?}
	Like any other TTRPG, there is a ST and the amount of players they can comfortably handle (typically 2-4). The group will create a collaborative story using Original Characters (OCs) while the ST throttles tension with various Challenges involving Fudge dice.
	
	\begin{description}
		\item[Note] this specific zine will focus on traditional combat based challenges and scenarios, but keep in mind that any activity can be a Story of Adventure!  You can do horse racing, competitive cooking, etc. Just be prepared for it to get weird :)
	\end{description}
	
	\subsection*{What are Fudge dice?}
	Fudge dice (dF) are six sided dice (d6) that have two sides with a plus, two sides with a minus, and two sides with a blank. A typical roll is 4dF where [+] and [-] cancel each other out. The result skews Neutral. These dice are used to visually identify degrees of success and failure at a glance. 
	
	If all you have are d6s, then you can treat them like d3 in which:
	\begin{itemize}
		\item 1 and 2 is [-]
		\item 3 and 4 is [\_]
		\item 5 and 6 is [+]
	\end{itemize}
	
	\subsection*{Challenges and Overcoming Them}
	\begin{description}
		\item[Chance] \textit{standard 4dF roll with no external modifiers.} These can be used to decide the contents of a chest, an NPC's initial disposition, or even the weather.
		\begin{description}
			\item[Note] A total of 0+ is a result that neither benefits nor harms the party. However, that doesn't necessarily imply there are no consequences.
		\end{description}
		\item[Challenge] \textit{x+ needed to pass.} These are the same as Chance rolls except with a greater likelihood for failure. 
		\item[Cumulative] \textit{x+ intended to be solved over multiple turns}. 
		\item[Competitive] \textit{xdF vs xdF}
	\end{description}
		
	\section*{Banking dice}
	Although the system is designed around degrees of success, the finesse of the Game Master rests in controlling tension. There are a couple different ways to perform a simple skill check:
	\begin{itemize}
		\item Threshold: Count [+] needed to pass. Ignore other results.
		\item Empowered: [\_] count towards threshold. Ignore [-] results.
	\end{itemize}
	
	%\section*{Campaigns}
	
	%% CONTENT ENDS
	
	% Back cover
	
	\newpage
	
	Reverse this zine for character creation
	\\jacob@jhfreedman.com
	
\end{document}