%% Rules for formatting inserts

%%% fillrline 1: Percentage of line width
%%% Inserts a solid line to partition off parts of text
\def\fillrline#1{%
	\vspace{-1em}
	\begin{figure}[H]
		\begin{center}
			\makebox[#1]{\hrulefill}
		\end{center}
		\vspace	{-1.25em}
	\end{figure}
	}
%%% fillrlinemid 60%
%%% fillrlinemost 80%
%%% fillrlinefull 100%
%%% Quick formats for filler lines at different lengths
\def\fillrlinemid{\fillrline{0.6\linewidth}}
\def\fillrlinemost{\fillrline{0.8\linewidth}}
\def\fillrlinefull{\fillrline{1.0\linewidth}} % :w for WHOOPS

%%% newplayr 1: Text
%%% Sets up a figure box for callouts for new players
\def\newplayr#1{%
        \begin{figure}[H]{%
                        \makebox[3em]{\dotfill}\ {\large New Players!}\ \dotfill\vspace{-2em}\linebreak
                        \begin{flushleft}
                                #1\vspace{-1em}
                        \end{flushleft}
                        \strut\dotfill
                        }   
        \end{figure}
        }   

%%% notetoGM 1: Text
%%% Sets up a callout box for would-be GMs
\def\notetoGM#1{%
        \begin{figure}[H]{%
                        \makebox[3em]{\hrulefill}\ {\large A Note to the GM}\ \hrulefill\vspace{-2em}\linebreak
                        \begin{flushleft}
                                #1\vspace{-1em}
                        \end{flushleft}
                        \strut\hrulefill
                        }
        \end{figure}
	}

%%% equview #1
%%% Makes the equations in the attribute section stand out. Breaks lines and makes the text big and centered
\def\equview#1{%
	\begin{figure}[H]
		\vspace{-0.75em}
		\begin{center}
			{\large #1}\vspace{-1.00em}
		\end{center}
        \end{figure}
}


%%% footnoteref 1: label
%%% Adds a page number reference in the foot notes based on the passed label.
\def\footnoteref#1{\footnote{\nameref{#1}, pg. \pageref{#1}}}

%% Attribute Section Formatting ----------
%%% calloutline 1: callout ID 2: text
%%% Adds a small callout to the beginning of the line for ease-of-reading
\def\calloutline#1#2{\noindent\vbox{\hbox{\vdots\ {\large #1} \vdots}}\leavevmode\linebreak#2\vspace{1.00em}\linebreak}

%%Attribute Intro
\def\attrlineIntro#1{\begin{center}\textsl{#1}\end{center}}

%% Attribute Line "How" 
\def\attrlineHow#1{\calloutline{How it works}{#1}}
%% Attribute Line "Where"
\def\attrlineWhere#1{\calloutline{Where it's used}{#1}}

%%% vsObstacle/vsCharacter 1: Text
%%% Adds a small callout to the beginning of the line to better segregate conditions for each situation
\def\vsObstacle#1{\calloutline{vs Obstacle}{#1}}
\def\vsCharacter#1{\calloutline{vs Character}{#1}}

%%% burstline 1: Text
%%% Adds a small callout to the beginning of the line to make the BURST boost seem more important and exciting
\def\burstline#1{\calloutline{\BURST}{#1}}

%%% Repeater 1: Number of times 2: Text to repeat
%%% Repeats #2 #1 number of times 
\def\repeater#1#2{
	\newcount\tmp
	\tmp=#1
	\loop
	#2
	\advance \tmp -1
	\ifnum \tmp>0
	\repeat
}

%%%% For testing purposes only 
%%% Sample
%%% Makes a paragraph with sample text
\def\sampletxt{
Hey this is just a block of text to check the formatting because just doing lipsum is boring and I can't think of anything else at the moment for this section so we should brainstrom some stuff later to think of stuff to put in these sections because I can't think of anything else at the moment for this section so we should brainstrom some stuff later to think of stuff to put in these sections because I can't think of anything else at the moment for this section so we should brainstrom some stuff later to think of stuff to put in these sections because I can't think of anything else at the moment for this section.}
