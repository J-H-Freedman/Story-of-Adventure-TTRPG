
\part{Character Creation}

\chapter{Character Attributes}

\section{\attribute\ Points}

Your character's actions and abilities are built around 5 \attribute\ 
categories: \POWEful, \KNOWful, \ENDUful, \SPEDful, and \FOREful.Throughout
your campaign(s) you'll be able to build in these areas by spending
\attribute\ points (\attrval). Each \attribute\  gives a purely
bonus-based benefit, which means that it's fine, and expected, for
your character to have 0's in some of their \attribute\  categories.
Thematically a 0 just means that your character is average in that
area; you will still be able to make all of your necessary rolls--
you just won't get any bonuses! 

\newplayr{When making a new character you start with 50\attrval\ to spend between each \attribute . However! This means your starting \HPful\ is 50 and your starting \ENful\ is 100 so be careful!}

\section{\BURST\index{BURST}}

When an \attribute\  reaches \textbf{100}, it triggers a \BURST\ !
This does a few things:
\begin{itemize}
\item Grants a permanent upgrade specific to the \attribute\ that triggered the \BURST.
\item Grants a new \techn\  tied to the \attribute.
\item Reduces the \attribute\  back down to \textbf{10}.
\end{itemize}
\newplayr{Despite the \attribute\ reduction, the upgrades from a \BURST\ are substantial and I recommend triggering a \BURST\ whenever possible. However! Moderation is important. Triggering too many at once could leave you in a tight spot so I also recommend keeping a few \attribute\ scores high until the others have recovered.}

\section{\attribute\ Usage}

\subsection*{Notation}
\begin{itemize}
\item d100\index{d100} refers to a number from 1 -- 100. This can be from
either two ten-sided dice rolled to create a number with a range of
100, a random number generator, or however you decide to generate
it.
\item XXX refers to the attribute ``XXX'' and everything that entails.
\item {[}XXX{]} refers to the variable ``XXX'' itself.
\end{itemize}

\subsection{\KNOWful\  {[}\KNOW\index{Attribute!KNO}{]}}

\emph{$d100+[\KNOW]$}

\noindent For \KNOWful\  checks you want your total to be \textit{above}
a target amount.

\noindent Roll a d100 and add {[}\KNOW{]} to the amount. If it's
above the target then you pass, and if not then you fail.

\noindent Any time you would ask, ``Does my character know/notice...''
you may roll a \KNOW.

\noindent In addition, {[}\KNOW{]} is added to your \skills\  checks.\footnote{Skill checks, x.x.x}

\noindent Upon a \BURST, allocate 100 \skills\  points into your
existing \skills\ (s) however you please.\footnote{skill points, x.x.x}

\subsection{\POWEful\  {[}\POWE\index{Attribute!POW}{]}}

\emph{$d100_{1}+d100_{2}+2([\POWE])$}

\noindent For \POWEful\  checks you want to try to roll the \textit{biggest},
and we mean BIGGEST, total possible.

\noindent Roll 2 d100s and add {[}\POWE{]} to each result

\noindent A \POWE\  may be utilized for most Double-Actions

\noindent Upon a \BURST, permanently increase the \POWE\ multiplier
by 1 (e.g. $2\rightarrow3$).

\subsection{\ENDUful\  {[}\ENDU\index{Attribute!END}{]}}

\emph{$(d100\pm[\ENDU])\divideontimes$ turns until pass or fail}

\noindent For \ENDUful\  checks your rolls add to a running total
proceeding towards specific\textit{ thresholds}.

\noindent Every turn of an \ENDU\  check begins with a d100 roll,
but it proceeds differently depending on if you're \textit{storing}
or \textit{spending} an amount:
\begin{itemize}
\item If the amount is being \emph{stored} to your total for the next turn(s),
add {[}\ENDU{]} to the roll and the result to the total. The desired
effect occurs \emph{when} the \textit{storing} threhsold is reached.
\item If the amount is being \emph{spent} from your total from the previous
turn(s), subtract {[}\ENDU{]} from the roll and the result from the
total. The desired effect occurs \emph{until} the \textit{spending}
threshold is reached.
\end{itemize}
\noindent Any time your action would take multiple turns, you may
run an \ENDU. Depending on context, an \ENDU\  may either be a basic
action or a double-action.

\noindent Upon a \BURST, permanently increase the \emph{turn }multiplier
by 1. This means you roll an additional action for \emph{storing}
and skip an additional action for \emph{spending.}

\noindent \newplayr{Generally speaking, storing to your total will produce positive effects in later turns so you'll want to 
try to build this as quickly as possible. Spending from your total will also block undesirable effects 
so you'll want to keep this as low as possible.}

\subsection{\SPEDful\  {[}\SPED\index{Attribute!SPD}{]}}

\emph{total $(d100+[\SPED])$ until pass or fail}

\noindent For \SPEDful\  checks you'll roll for successes against
an ever-building threshold until you reach the target \textit{number
of successes} or you \textit{fail}.

\noindent Roll a d100 and add {[}\SPED{]}. If the result is above
the target then you succeed. If you succeed, start a running total
with the result, add 10 to the target threshold, and repeat the process
until either condition is met. With every subsequent success you'll
add the result to the running total and add 10 to the threshold. 

\noindent Any time your action encompasses multiple smaller actions,
you may roll a \SPED.

\noindent Upon a \BURST, permanently gain an addtional free reroll
upon \SPED\  failures.

\subsection{\FOREful\  {[}\FORE\index{Attribute!ESP}{]}}

\emph{guess within $\pm[\FORE]$ against the opposing roll}

\noindent For \FOREful\  checks you're guessing what the opposing
roll will be within a buffer of $\frac{1}{2}$ your \FORE .

\noindent Before an opposing roll, you may conduct \FORE. If the
roll is within the range of $(guess-\frac{1}{2}[\FORE])$and $(guess+\frac{1}{2}[\FORE])$,
you succeed.

\noindent Any time your character attempts to predict or react to
external stimuli, roll \FORE.

\noindent Upon a \BURST, pick an additional number to guess.

\chapter{Abilities\index{Abilities}: \skills s and \techn s}

Abitites are extra modifiers for your \attribute s during various
game situations. Generally they work by replacing $[XXX]$ with $([XXX]+[Ability])$
in XXXs' formula. There are 3 categories for abilities: \skills s, \techn s, and \AWEAWF.

\section{\skills s\index{Skills}}

\skills s represent your general knowledge of things, and provide
bonus(es) specifically to your \KNOW  challenges. Their bonus is
directly equal to the amount of \skills Points (\skillval ) allocated
to them. They affect the formula thusly:

$d100+([\KNOW]+[\skills])$

So if you have a {[}\KNOW{]} of 20 and a Knitting \skills\  of 37;
then in a Knitting challenge, you will add your d100 roll to \textbf{57
}instead of 20 to calculate the result.

\newplayr{\skills s are used to perform a variety of quick tasks and can flesh out your character's past, present, and future. Before using \techn s, make sure you understand \skills s first.}

\section{\techn s\index{Techniques}}

\techn s are advanced maneuvers representing specific uses of \attribute s
during challenges. Unlike \skills s these are not automatic boosts
and cost \ENful\ to use; starting at 10 and increasing by 5 per \advancement\ (\advanmini). 

Initially the calculations will proceed much like \skills s with
the formula changing to $([technique]+[attribute])$ in the place
of $[attribute]$, however as they are upgraded with \advanmini\ they
can receive new effects.

\newplayr{Typically \techn s will be used in character vs character challenges, but they can be used wherever applicable-- even in normal challenges!}

\section{\AWEAWF\index{AWFUL} \index{AWESOME}}

To make your campaign\footnote{campaigns, x.x.x} more interesting,
I recommend using \AWEAWF (A/A) abilities. This creates an additional
ability with the value of +50 \emph{and} an additional ability with
the value of -50\marginpar{There may be situations in which your Skill or technique is a negative
number. This acts the same as when it is a positive number but you
subtract instead of add. You are, afterall, adding a negative number}.

This may seem like a free bonus, but I recommend the player and/or
the GM put the character in AWEFUL situations to make the game interesting.
When an AWEFUL ability is rolled, store 1 Cool token.\footnote{Cool, x.x.x}

\chapter{Character Development}

\section{Advancement}

At the end of a session, the GM may give players the opportunity to
advance one \attribute and/or ability

\subsection{Attribute Advancement}

When given the opportunity, you pick an attribute you used during
this session. To advance it, roll a d10\marginpar{A d10 is a ten sided die}.
The results are the amout of AP the attribute is increased by.

\subsection{Ability Advancement}

When given the opportunity, you may choose to either advance an existing
ability or add a new one.

\section{Injuries \& Enhancements}

You will experience physical, mental, and spiritual changes on your
journey. I recommend keeping track of these over the campaign

\section{Possessions}

You will acquire a lot of artifacts, equipment, and trinkets over
your journey. (author's note: right now it's all one area. I'm considering
segmenting based on playtesting)

\section{Persons, Places, \& Things of Interest}
