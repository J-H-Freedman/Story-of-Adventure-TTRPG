% Don't compile this! This is only part of the manual it won't work on its own.
\part{Character Creation}\label{part:char_creation}
\chapter{Character \attribute s}\label{ch:char_attributes}

% Intro --------------------------
\section{\attribute s}\label{sec:attributes}
Your character's actions and abilities are built around 5 \attribute\ 
categories: \POWEful, \KNOWful, \ENDUful, \SPEDful, and \FOREful. Each \attribute\  gives a purely
bonus-based benefit, which means that it's fine, and expected, for
your character to have 0's in some of their \attribute\  categories.
Thematically a 0 just means that your character will be average in that
area; you will still be able to make all of your necessary rolls--
you just won't get any bonuses! 

\subsection{\attribute\ Enhancement and \attrval}\label{subsec:attr_points}
Throughout your campaign(s) your \attribute s will be able to be enhanced by spending \attribute\ Points (\attrval). Each point spent will increase the \attribute\ directly by the same amount. New characters will start with 50 \attrval\ to allocate how they wish. Afterwards they'll be earned at the end of a session, or by some other alternate means.

\newplayr{Since new characters start with 50 \attrval\ this also means their starting \HPful\ is 50 and their starting \ENful\ is 100 so be careful!}

% Burst --------------------------
\subsection{\BURST\index{\BURST}}\label{subsec:BURST}
When an \attribute\ reaches \textbf{100}, it triggers a \BURST !
This surge of power permanently upgrades the character's proficiency with that \attribute , but the shock also reduce that \attribute\ down to \textbf{10}.
\newplayr{Moderation is important! A \BURST\ can be a powerful upgrade, but triggering too many at once could leave you in a tight spot. I recommend keeping a few \attribute\ scores high until the others have recovered from the shock.}

% Start Attribute Usage -------------------------
\section{\attribute\ Usage}\label{sec:attr_usage}
Your character's \attribute s are the core of everything that they do in \gametitle. Most, and possibly all, of your character's actions will involve forming a total using 1 or more \attribute . Each \attribute\ has its own method of being calculated as well as conditions for passing and failing\footnoteref{sec:success}. 

\subsection*{Usage Notation}\label{subsec:notation}
The notation for this section is as follows:
\begin{itemize}
	\item d100\index{d100} refers to a number from 1 to 100. This can be from either two ten-sided dice rolled to create a number with a range of 100, a random number generator, or however you decide to generate it.
	\item \textbf{XXX} refers to the \attribute\ 's process and everything that it entails.
	\item {[}XXX{]} refers to the value of the \attribute\ itself.
\end{itemize}
\clearpage

% Knowledge ------------------------
\subsection{\KNOWful\ {[}\KNOW\index{\attribute!\KNOWful}{]}}\label{subsec:kno}
\equview{$d100 + [\KNOW] + \skill $}
\attrlineIntro{Use all of the \skill s that you know to form a total}\attrlineHow{Roll a d100 and add {[}\KNOW{]} as well as the scores of \textbf{all} \skill s\footnoteref{sec:skills} applicable to the challenge. If your total is above the target threshold then you pass, otherwise you fail.}
\attrlineWhere{Any time your character attempts to understand something or you would otherwise ask, ``Does my character know/notice...'' you may execute a \textbf{\KNOW} action.}
\burstline{Unlike the other \attribute s, \KNOW does not scale directly with your \BURST\ rank. Instead, after a \BURST\ is triggered, add 100 \skill\ Points\footnoteref{sec:skills} into your existing \skill (s) however you wish.}
\clearpage

% Power ----------------------------
\subsection{\POWEful\ {[}\POWE\index{\attribute!\POWEful}{]}}\label{subsec:pow}
\equview{$(d100_{1} + {[}\POWE{]}) + (d100_{2} + {[}\POWE{]}) + ...$}
\attrlineIntro{Break through by building the biggest total possible from a set amount of rolls}
\attrlineHow{Roll 2 d100s, add {[}\POWE{]} to each result, and then add them together.}
\attrlineWhere{Any time your character attempts an above average feat of strength (e.g. striking, imparting a telekinetic force, lifting something) you may execute a \textbf{\POWE} action.}
\burstline{Roll a number of additional d100s equal to your \BURST\ rank. Add {[}\POWE{]} to each roll and add them to your total.}
\clearpage

% Endurance-------------------------
\subsection{\ENDUful\ {[}\ENDU\index{\attribute!\ENDUful}{]}}\label{subsec:end}
\equview{$(d100 + [\ENDU]) + Target\ d100_1 + Target\ d100_2$}
\attrlineIntro{Brace and mitigate the opposing total by absorbing rolls}
\attrlineHow{Roll a d100 and add your {[}\ENDU{]}. Then take 2 of the lowest opposing d100 rolls and add them to your total.} 
\attrlineWhere{Any time that your character needs to brace themselves against something, focus, or otherwise endure the effects of something you may execute a \textbf{\ENDU} action.} 
\burstline{Take an additional number of d100s equal to your \BURST\ rank from the opposing total.}
\clearpage

% Speed ----------------------------------
\subsection{\SPEDful\ {[}\SPED\index{\attribute!\SPEDful}{]}}\label{subsec:spd}
\equview{total $(d100 + [\SPED])$ until pass or fail}
\attrlineIntro{Throw caution to the wind and blitz forward to build a total with as many rolls as you can make}
\attrlineHow{Before rolling starts a threshold is determined by the GM. Afterwards roll 2 d100s, and for every roll that is above the threshold, add that roll and your {[}\SPED{]} to the total. Any d100s that are below the threshold are removed, and then the process repeats with the remaining d100s. The process is repeated until a target threshold is reached, or you run out of d100s.}
\attrlineWhere{Any time your character is attempting to do something abnormally fast (e.g. a quick getaway, a rapid series of blows, hastily typing something) you may execute a \textbf{\SPED} action.} 
\burstline{Roll an additional number of d100s equal to your \BURST\ rank.}
\clearpage

% Foresight ------------------------------
\subsection{\FOREful\ {[}\FORE\index{\attribute!\FOREful}{]}}\label{subsec:esp}
\equview{guess within $[\FORE]$ against the opposing roll}
\attrlineIntro{Predict what an opposing total will be to reverse a portion of it back to the user}

\attrlineHow{Before an opposing action, you declare what you predict the resulting total will be. You don't roll at all, and your starting total is what you guess. If the distance between your guess, and their total is less than your {[}\FORE{]} then you may steal their highest roll and add it to your total.\linebreak In \textsl{Character vs Obstacle}\footnoteref{sec:vs_obstacle} challenges the GM will announce an amount of d100s that you will need to guess the sum of.\linebreak In \textsl{Character vs Character}\footnoteref{sec:vs_character} challenges you will need to guess what the total of their action will be.}
\attrlineWhere{Any time your character is using their intuition to sense or predict something (e.g. where an attack is coming from, breaking something's disguise, the weather) you may execute an \textbf{\FORE} action.}

\burstline{Increase the number of guesses by your \BURST\ rank, and steal the next highest d100 roll per successful guess. Your starting total is the highest of your guesses.}
\clearpage
% Start Abilities
\chapter{Abilities\index{Abilities}: \skill s and \techn s}\label{ch:skills_and_tech}
Abitites are extra modifiers for your \attribute s during various situations. Generally they will work by replacing $[XXX]$ with $([XXX]+[Ability])$ in the usage formula. There are 3 categories for abilities: \skill s, \techn s, and \AWEAWF.

\section{\skill s\index{\skill s}}\label{sec:skills}
\skill s represent your general knowledge of all things, and provide bonus(es) specifically to your \KNOW\ challenges. Their bonus is directly equal to the amount of \skill\ Points (\skillval) allocated to them. They affect the formula as so:
\equview{$d100 + ([\KNOW]+[\skill])$}
As an example, if your character has a {[}\KNOW{]} of 20 and a Knitting \skill\ of 37; then in a Knitting challenge, you will add your d100 roll to \textbf{57} instead of 20 to calculate the result.
\newplayr{\skill s are used to perform a variety of quick tasks and can flesh out your character's past, present, and future. Before using \techn s, make sure you understand \skill s first.}

\section{\techn s\index{\techn}}\label{sec:techniques}
\techn s are advanced maneuvers representing specific uses of \attribute s during challenges. Each \techn\ is tied to a specific \attribute\ action, and the actions are processed as normal. However each \techn\ also has its own rules to follow, and will add new conditional boosts or effects depending on various factors of your action. Unlike \skill s these are not automatic boosts and cost \ENful\ to use; starting at 10 \EN\ and increasing by 5 \EN\ per \advancement\ (\advanmini). 

\newplayr{Typically \techn s will generally be used in character vs character challenges, but they can be used wherever applicable-- even in obstacle challenges!}

%Awesome / Awful 
\section{\AWEAWF\index{AWESOME}\index{AWFUL}}\label{sec:aweawf}
\AWEAWF\ (A/A) abilities represent a character's fringe expertise-- the areas that they are the best and the worst in. \AWEAWF\ abilites are action-wide scenario boosts. In an action where the character's \AWE\ ability is applicable add \emph{50} to that \attribute , and conversely subtract \emph{50} when an \AWF\ ability is applicable. Due to the increased risk, when an \AWF\ ability is rolled the player will earn 1 Cool Token.\footnoteref{sec:cool_tokens}. 

\newplayr{There may be situations in which your \skill\ or \techn\ is a negative number. This acts the same as when it is a positive number but you subtract instead of add. You are, afterall, adding a negative number.}

\notetoGM{This may seem like a free bonus, but to make your campaign(s)\footnoteref{subsec:campaigns} more interesting I recommend you, or even the player, try to put the character in \AWEAWF\ situations.}

% Character Development ------------------------------------------------------
\chapter{Character Development}\label{ch:char_develop}
Characters will always be developing as they're used and can undergo lasting changes at the end of, or even sometimes during, a session.

\section{\advancement\index{\advancement}}\label{sec:advancement}
The GM may give players the opportunity to advance one \attribute\ and/or ability

\subsection{Earning \attribute\ Points}\label{subsec:adv_techniques}
When given the opportunity, you can pick an \attribute\ you used during this session and roll a d10. The \attribute\ is increased by a number of \attrval\ equal to the number rolled.

\subsection{Ability \advancement}\label{subsec:adv_ability}
When given the opportunity, you may choose to either advance an existing ability or add a new one.

\section{Injuries \& Enhancements}\label{sec:adv_injuries}
You will experience physical, mental, and spiritual changes on your journey. I recommend keeping track of these over the campaign.

\section{Possessions}\label{sec:possessions}
You will acquire a lot of artifacts, equipment, and trinkets over your journey. (author's note: right now it's all one area. I'm considering segmenting based on playtesting)(editor's note: we may need to reorganize a bit at some point yes)

\section{Persons, Places, \& Things of Interest}\label{sec:interests}
	