%% Don't compile this on it's own! This is only part of the manual it won't work
\part{Basic Mechanics}\label{part:basics}
\chapter{Challenges\index{Challenges}}\label{ch:challenges}
Passing challenges are the core to every game, let alone role playing games. 

In \gametitlemini, you will almost always want acquire numbers as
high as possible to pass these challenges. You will have plenty tools
to help you in this goal: dice rolls, attribute effects \& modifiers,
and skills/techniques. 

\section{The Total}\label{sec:thetotal}
Challenges are vast and varied in the world of \gametitlemini\ , but regardless of whether you're fighting a mobster demi-god, sweaving around asteroid-belt ring worms, or painting a fence your outcome proceeds based on \emph{The Total}. \emph{The Total} the is result of all possible rolls, modifiers, and any other ancillary variables that have some influence in the situation at hand. In general it is an overall measurement of how well you did given the circumstances of the challenge, and influences everything from baseline success to damage calculation.

\subsection{Determined Threshold\index{Determined Threshold}}\label{subsec:determined_threshold}
The \emph{Determined Threshold}, sometimes just referred as the \emph{Threshold}, is a baseline that must be met or surpassed for success in the given challenge. In general the higher the number, the more difficult the challenge is both thematically and mechanically. There are many \emph{Threshold}s variations, but they will always be a measurement of something within the challenge such the number of successes, the overall total, or even the number of failures. Depending on the challenge faced the \emph{Threshold} can be set by the GM or by another character's actions.

\newplayr{The GM may or may not tell you what exactly this threshold is.}
\notetoGM{Sharing this information may create tangable tension, but hiding it should be flat out scary. Pick your moments well.}


\section{Success, Pass, \& Fail}\label{sec:success}
An important semantic to define is \emph{Succeed} vs \emph{Pass}:

\begin{itemize}
\item Success\index{Challenges!Success} \vdots\ One instance of the requirements of the Challenge has been met.
\item Pass\index{Challenges!Pass} \vdots\ The Challenge is completed and the result is in your favor.
\item Fail\index{Challenges!Fail} \vdots\ The Challenge is completed and the result is not in your favor.
\end{itemize}
In most Challenges meeting the \emph{Threshold} requirements will allow you to pass, but sometimes the overall result will also be altered by the \emph{Degree} between the totals of those taking part.   

\notetoGM{Not every fail needs to result in disaster. Neutral results can also occur when you try to do something and fail.}


\subsection{Degree of Success and Failure (DSF)\index{Degree}}\label{subsec:dsf}
In all cases that do not result in a tie (i.e. most of them) you will
pass or fail by an amount referred to as its \emph{Degree}. 

\equview{$Initiator Total - Defender Total = Degree$}

\emph{Degrees} can be used intuitively to interpret specific results from
a binary pass or fail state, and can also be used as a variable to be used in further calculations
(e.g. Damage\footnoteref{subsec:damage_calc}).

\subsection{Damage\index{Damage} Calculation}\label{subsec:damage_calc}
If a challenge deals \emph{damage} it is calculated by the DSF\footnoteref{subsec:dsf} from that challenge. Once calculated, it will then be distributed to the receiver's \HPful\ and \ENful\ as necessary. Players will generally be able to distribute their damage between \HPful\ and \ENful\ as they see fit, however some situations may force all of the damage onto one or the other without compromise.\footnoteref{subsec:damage_types} 

\subsection{In Case of a Tie\index{Challenges!Tie}}\label{subsec:tie}
A tie. An event that is only spoken of through whispers. The appearance of a realm of sheer impossibility in which neither success nor failure can exist. The brief moment that it breaches the mortal plane is enough to send ripples throughout the galaxy, shifting environments and worldviews alike. It is a curiously benevolent impossibility, as all those who witnessed a tie found themselves receiving 1 Cool Token\footnoteref{sec:cool_tokens}.

% Challenges
\section{Character vs Obstacle\index{Challenges!Obstacle Challenge}}\label{sec:vs_obstacle}
\emph{Obstacle Challenges} are initiated between your character and something that is relatively non-sentient. These are generally the most basic challenge that will be encountered, and in most cases will be decided by whether the applicable \attribute\ check passes or fails. The threshold, total, and any other applicable number for an \emph{Obstacle Challenge} is determined directly by the GM.

% Character VS Character Challenge
\section{Character vs Character}\label{sec:vs_character}
\emph{Character Challenges} are initiated between one to many characters. They can be more involved than \emph{Obstacle Challenges}, and will generally take place during encounters. In \emph{Character Challenges} one character acts as an initiator and proceeds with an \attribute\ check to form a total and/or threshold that the other characters need to meet or defend against with an \attribute\ check of their own. 

\section{Cool Tokens\index{Cool Tokens}}\label{sec:cool_tokens}
\emph{Cool Tokens} are an elusive trinket that, when exchanged, grant you a \emph{alternate} d100 to be used in your upcoming challenge. Both d100s will be rolled for that challenge, and you'll be able to alter the results by choosing which d100 you want to use, or even both! The boons offered by Cool Tokens are great, but they have strict guidelines for usage. To spend a Cool Token you must announce it to the GM, and you also must not have started rolling for that challenge.

Only the most daring will be able to earn Cool Tokens. While their awardance is at the GM's digression, generally those who make creative and...riskier plays will find themselves with a Cool Token in hand. It ultimately is a reward for players who think outside the box, and can be given by the GM when they feel like it will make the story better.

It may build character; it may be narratively interesting; it may be a character flaw; it may be a failed roll during high tension; It could be anything, but it must be cool.

\notetoGM{It is recommended for the maximum Cool Token cap to be 1 per player,
but this is just a suggestion and ultimately left up to you to decide in your campaign.}
\newplayr{"Cool" is subjective, but this does not excuse stupidity. Never assume the GM will give you Cool Tokens. Roleplay appropriately and see what happens.}

\chapter{\HPful\ and \ENful}\label{ch:hp_and_en}
\section{\HPful\index{\HPful}}\label{sec:hp}
\HPful\ is a crucial metric that represents tangable, lasting damage to a character's body. Too much \HPful\ damage could result in a
\textit{permanent wound.} Your \emph{Maximum \HPful} is equal to your total spent \attrval\footnoteref{subsec:attr_points}.

\newplayr{Keep your guard up! Although most damage can get absorbed by energy, some situations may utilize \emph{Direct Damage}\index{Direct Damage} which can force all of the incoming damage to \HPful !}

\subsection{Damage Types and Diversion\index{Diversion}\index{Damage}}\label{subsec:damage_types}
Damage can be delivered in two forms:
\begin{itemize}
	\item Normal Damage \vdots\ Threatening, but mitigatable. Normal damage is able to be shielded by \ENful\ reserves, and can be allocated between \HPful\ and \ENful\ as \emph{the receiver} sees fit.
	\item Direct Damage \vdots\ Terrifying, and very much life-threatening. This type bypasses all defenses and is delivered exclusively to \HPful .\index{Direct Damage}
\end{itemize}

\subsection{Permanent Wounds\index{Permanant Wounds}}\label{subsec:wounds}
\emph{Permanent Wounds} are lasting injuries or conditions that will be with your character throughout their campaign(s).

\subsection{Death\index{Death}}\label{subsec:death}
If a character's \HPful\ falls to or below 0, the character is dead. The good news is that death doesn't have to be permanent!
This may be an opportunity for your friends to go on an adventure to revive you while you train in the afterlife.

However, if the player desires, they may instead create a new character
to be recruited by the adventuring party. Just know that it is a \emph{choice}
and not a neccessity.

\section{\ENful\index{\ENful}}\label{sec:energy}
\ENful\ is a critical resource within \gametitlemini\ with a variety of uses from executing \techn s\footnoteref{sec:techniques} to mitigating critical injury.
Managing how and when to use your \ENful\ is vital to survival. Your \emph{Maximum \ENful} is equal to double your \emph{Maximum \HPful}.

\subsection{\techn\ Resources}\index{\techn}\label{subsec:tech_resources}
To use a \techn\footnoteref{sec:techniques}, an \ENful\ cost must be paid. The starting cost is 10 \EN\ and with each level of \advancement\footnoteref{sec:advancement}\index{\advancement}, the cost increases by 5 \EN.

\subsection{Overexertion}\label{subsec:exhaustion}
When a character's \ENful\ falls to to 0, the character is overexerting themself. All instances that would normally go to \ENful\ instead go to \HPful . You may choose to pass out at any time when overexerting yourself-- even if there are challenges you are currently participating in.

\subsection{Passing Out\index{Passing Out}}\label{subsec:pass_out}
When you pass out, you may not act in any way until rested or healed.
\newplayr{It is \emph{universally} frowned upon to attack a passed out opponent, and
only the most evil character would spitefully do such an act.}

\chapter{Encounters}\label{ch:encounter}
Encounters are major, generally combat focused, events. The structure of how events take place are usually unimportant but are still expected to occur frequently. Encounters are organized as such:
\begin{enumerate}
	\item {\large Encounter}\index{Encounter} \linebreak\
	This encapsulates the entire situation. \emph{Encounters} proceed over a predetermined or indefinite number of engagements known as \emph{Rounds}. An encounter ends when: 
	\begin{itemize}
		\item The predetermined Round countdown reaches 0
		\item Only one team is able to take Turns
	\end{itemize}
\item {\large Rounds}\index{Round} \linebreak\
	These encapsulate \emph{every} character's Turn. After every character has had their Turn, a new \emph{Round} begins.
\item {\large Turns}\index{Turn} \linebreak\
	These encapsulate the \emph{selected} character's Actions. In most instances each character takes one \emph{Turn} per Round
\item {\large Actions}\index{Action} \linebreak\
	These are how characters actually do things. In most instances each character will take two \emph{Actions} per Turn.
\end{enumerate}

\section{Actions}\label{sec:actions}
Character \emph{Actions} encompass everything that they can do during encounters. Each character will generally only be able to take \emph{two} actions during their turn so it is important to know how they differ. Actions are as follows:
\begin{itemize}
	\item An \attribute\ check vs a character or obstacle within the encounter location. If the target is a character, they may execute a response \attribute\ check where applicable\footnoteref{sec:attr_usage}.
	\item Charge an \attribute\ to temporarily increase its \BURST\ rank for the next action.
	\item Execute a \techn\footnoteref{sec:techniques}. The target may perform a response \attribute\ check depending on the context.
\end{itemize}