
\part{Basic Mechanics}

\chapter{Challenges\index{Challenges}}

Passing challenges are the core to every game, let alone role playing
games. 

In \gametitlemini, you will almost always want aquire numbers as
high as possible to pass these challenges. You will have plenty tools
to help you in this goal: dice rolls, attribute effects \& modifiers,
and skills/techniques. 

\section{Success, Pass, \& Fail}

An important semantic to define is \emph{Succeed} vs \emph{Pass}:
\begin{itemize}
\item Success\index{Challenges!Success} The total\marginpar{``The total'' refers to the final number tested against the Challenge.
This comes after possible rolls, modifiers, and other variables.} meets one instance of the requirements of the Challenge.
\item Pass\index{Challenges!Pass} The Challenge itself is completed and
the result is positive.
\item Fail\index{Challenges!Fail} The total does \emph{not} meet the requirements
of the challenge and the result is neutral\marginpar{Neutral results occur when you try to do something and fail. Not every
fail needs to result in disaster.} or negative.
\end{itemize}
Most of the Challenges you will encounter require only one success
to pass. However some challenges will require multiple successes or
will result in a binary succeed/fail instead of counting the degree.

\subsection{Degree of Success and Failure (DSF)\index{Degree}}

In all cases that do not result in a tie (i.e. most of them) you will
succeed or fail by an amount refered to as its ``degree.'' 

$highTotal-lowTotal=degree$

Degrees can be used intuitively to interpret specific results from
a binary pass/fail state.

Degrees can also be used as a variable to be used in further calculations
(i.e. damage\footnote{Damage, x.x.x}).

\subsection{In Case of a Tie\index{Challenges!Tie}}

The win of a tie \emph{always} belongs to the character conducting
the action\footnote{Actions, x.x.x}.

\section{Character vs Obstacle\index{Challenges!Obsticle Challenge}}

These are the most common kinds in TTRPGs. They are generally non-combat
related and can range from climbing a fence to painting a fence to
breaking a fence. 

These are \emph{Obstacle Challenges} and are tied to your Attributes.\footnote{Attributes, x.x.x}
These typically involve you rolling over a determined threashold decided
by the GM. They may be treated as either a pass/fail binary or be
treated with degrees of success or failure.

\subsection{Determined Threshold\index{Determined Threshold}}

When the GM requests that you complete an Obstacle Challenge, they
usually have a number in mind that must be met or surpassed. This
is the \emph{Determined Threshold}. The higher the number, the more
difficult the challenge is. Other variations can include a Determined
Threshold of Successes\index{Determined Threshold!of Successes} in
which you must succeed a number of times to pass.

\newplayr{The GM may or may not tell you what exactly this threshold is.}
\notetoGM{Sharing this information may create tangable tension, but hiding it should be flat out scary. Pick your moments well.}

\section{Character vs Character}

\subsection{Damage\index{Damage}}

Damage is calculated by the DSF\footnote{\emph{degrees}, x.x.x} from
the given challenge. Once calculated, the damage will then be distributed
to \HPful  and \ENful  as necessary. Players will generally be able
to distribute their damage between \HPful  and \ENful  as they see
fit, however some situations may force all of the damage onto one
or the other without compromise. \notetoGM{There are some contexts in which damage will go directly to health, but I recommend those to be few and far between.}

\section{Cool tokens\index{Cool tokens}}

Cool tokens are a special mechanic to encourage sub-optimal play for
the sake of gauranteeing successes later. This is to create creative
play that puts the player at an explicit disadvantage. It is given
by the GM when he feels like it will make the story better.

It may build character; it may be narratively interesting; it may
be a character flaw; it may be a failed roll during high tension;
etc.

Spending a cool token must be done and announced \emph{before} you
roll for your challenge. This is so the GM can hand you a \emph{second}
d100. For the result of your challenge; you may choose to use either
of the dice or both\marginpar{Usually you will chose to use both, but there are contexts in which
you will prefer only one number (i.e. ESP)}.

It is recommended for the maximum Cool token cap to be 1 per player,
but this suggested may be bent at the GM's discression

\newplayr{"Cool" is subjective, but this does not excuse stupidity. Never assume the GM will give you cool tokens. Roleplay appropriately and see what happens.}

\chapter{Health and Energy}

\section{\HPful\index{Health}}

\HPful\ is a crucial metric that represents tangable, lasting damage
to a character's body. Too much \HPful\ damage could result in a
\textit{permanent wound.}

\newplayr{Keep your guard up! Although most damage can get absorbed by energy, some situations may utilize {\slshape Direct Damage} which can force all of the incoming damage to \HPful !}.

\subsection{Maximum Health }

Maximum health is eaqual to the total spent Attribute Points\footnote{attrubutes points, x.x.x}

\subsection{Permanent Wounds\index{Perminant Wounds}}

\subsection{Death\index{Death}}

Once a character's health reaches or drops below 0, the character
is dead. The good news is that death doesn't have to be perminent!
This may be an opportunity for your friends to go on an adventure
to revive you while you train in the afterlife and learn a new technique.

However, if the player desires, they may instead create a new character
to be recruited by the adventuring party. Just know that it is a \emph{choice}
and not a neccessity.

\section{\ENful\index{Energy}}

\ENful\ is a critical resource within \gametitlemini\ with a variety
of uses from executing \techn s\  to mitigating critical injury.
Managing how and when to use your \ENful\ is vital to survival.

\subsection{Maximum Energy }

is equal to double your maximum health. \marginpar{Optional advanced rules complicate max energy calulations}

\subsection{Technique Resource}

To use a Technique\footnote{techniques, x.x.x}, an energy cost must
be paid. The base conse is 10. With each advancement\footnote{Advancement, x.x.x},
the cose increases by 5.

\subsection{Damage Diversion\index{Diversion}}

Except for Direct Damage, all other damage may be absorbed by energy
instead of health. The damage may be split at the defender's disgression.

\subsection{Exersion}

When a character's energy drops to 0, the character is exerting themself.
All instances that would normally go to energy instead go to health.
You may choose to pass out at any time when exerting yourself. If
there are no challenges you are currently participating in; or if
there are 

\subsection{Passing Out\index{Passing Out}}

When you pass out, you may not act in any way until rested or healed.
It is univerally frowned upon to attack a passed out opponent and
only the most evil character would spitefully do such an act.

\chapter{Encounters}

The structure of how events take place are usually unimportant but
are still expected to occur frequently. Hierchy is occurs as follows:
\begin{enumerate}
\item Encounter\index{Encounter}
\begin{itemize}
\item This encapsulates predetermined or indefinite rounds
\item An encounter ends when 
\begin{itemize}
\item (predetermined) the round countdown reaches 0
\item (indefinite) only one team is able to take Turns
\end{itemize}
\end{itemize}
\item Rounds\index{Round}
\begin{itemize}
\item These encapsulate \emph{every} character's Turn
\item After every character has had their turn, a new Round begins
\end{itemize}
\item Turns\index{Turn}
\begin{itemize}
\item These encapsulate \emph{the selected} character's Actions
\item Each character takes one Turn per Round
\end{itemize}
\item Actions\index{Action}
\begin{itemize}
\item These are when characters actually do things
\item Each character gets two Actions per Turn
\begin{itemize}
\item A Standard Action\index{Action!Standard} cannot be repeated multiple
times per Turn
\item A Double-Action\index{Action!Double-Action} both Actions in one Turn
\end{itemize}
\end{itemize}
\end{enumerate}

