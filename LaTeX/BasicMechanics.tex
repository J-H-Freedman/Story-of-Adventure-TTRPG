%% Don't compile this on it's own! This is only part of the manual it won't work
\part{Basic Mechanics}
\chapter{Challenges\index{Challenges}}
Passing challenges are the core to every game, let alone role playing games. 

In \gametitlemini, you will almost always want acquire numbers as
high as possible to pass these challenges. You will have plenty tools
to help you in this goal: dice rolls, attribute effects \& modifiers,
and skills/techniques. 

\section{The Total}
Challenge is as vast and varied in the world of \gametitlemini\ as the stars they're surrounded by, but regardless of whether you're fighting a mobster demi-god, sweaving around asteroid-belt ring worms, or painting a fence the outcome is  ruled by \emph{The Total}. This all-deciding number is result of all possible rolls, modifiers, and any other ancillary variables that have some influence in the situation at hand. Regardless of who you are, or think you are, how high or low \emph{The Total} is will determine if you pass or fail.

\section{Success, Pass, \& Fail}
An important semantic to define is \emph{Succeed} vs \emph{Pass}:

\begin{itemize}
\item Success\index{Challenges!Success} \vdots\ \emph{The Total} meets one instance of the requirements of the Challenge.
\item Pass\index{Challenges!Pass} \vdots\ The Challenge itself is completed and the result is positive.
\item Fail\index{Challenges!Fail} \vdots\ The total does \emph{not} meet the requirements of the challenge and the result is neutral or negative.
\end{itemize}
Most of the challenges you will encounter require only one success
to pass. However some challenges will require multiple successes or
will result in a binary succeed/fail instead of counting the degree.

\notetoGM{Neutral results occur when you try to do something and fail. Not every
fail needs to result in disaster.}

\subsection{Degree of Success and Failure (DSF)\index{Degree}}
In all cases that do not result in a tie (i.e. most of them) you will
succeed or fail by an amount refered to as its \emph{Degree}. 

$Higher Total - Lower Total = Degree$

\emph{Degrees} can be used intuitively to interpret specific results from
a binary pass or fail state, and can also be used as a variable to be used in further calculations
(e.g. Damage\footnote{Damage, x.x.x}).

\subsection{In Case of a Tie\index{Challenges!Tie}}
The win of a tie \emph{always} belongs to the character conducting
the action\footnote{Actions, x.x.x}.

% Obstacle Challenge
\section{Character vs Obstacle\index{Challenges!Obstacle Challenge}}
These are the most common kinds in TTRPGs. They are generally non-combat
related and can range from climbing a fence to painting a fence to
breaking a fence. 

These are \emph{Obstacle Challenges} and are tied to your \attribute s.\footnote{\attribute, x.x.x}
These typically involve you rolling over a determined threshold decided
by the GM. They may be treated as either a binary pass or fail or be
treated with degrees of success or failure.

\subsection{Determined Threshold\index{Determined Threshold}}
When the GM requests that you complete an Obstacle Challenge they
usually have a number in mind that must be met or surpassed. This
is the \emph{Determined Threshold}-- sometimes just refered as the \emph{Threshold}. The higher the number, the more
difficult the challenge is. Other variations can include a Determined
Threshold of Successes\index{Determined Threshold!of Successes} in
which you must succeed a number of times to pass.

\newplayr{The GM may or may not tell you what exactly this threshold is.}
\notetoGM{Sharing this information may create tangable tension, but hiding it should be flat out scary. Pick your moments well.}

% Character VS Character Challenge
\section{Character vs Character}
\subsection{Damage\index{Damage}}

\emph{Damage} is calculated by the DSF\footnote{\emph{Degrees}, x.x.x} from
the given challenge. Once calculated, it will then be distributed
to \HPful\ and \ENful\ as necessary. Players will generally be able
to distribute their damage between \HPful\ and \ENful\ as they see
fit, however some situations may force all of the damage onto one
or the other without compromise.
\notetoGM{\emph{Direct Damage}\index{Direct Damage} can create contexts in which the result will go directly to \HPful\, but I recommend those to be few and far between. Pick your moments. Make your players shake.}

\section{Cool Tokens\index{Cool Tokens}}

\emph{Cool Tokens} are a special mechanic to encourage sub-optimal play for
the sake of gauranteeing successes later. This is to create creative
play that puts the player at an explicit disadvantage. It is given
by the GM when he feels like it will make the story better.

It may build character; it may be narratively interesting; it may
be a character flaw; it may be a failed roll during high tension;
etc.

Spending a cool token must be done and announced \emph{before} you
roll for your challenge. This is so the GM can hand you a \emph{second}
d100. For the result of your challenge; you may choose to use either
of the dice or both. Usually you will chose to use both, but there are contexts in which you will prefer only one number (e.g. \FORE ).


\notetoGM{It is recommended for the maximum Cool Token cap to be 1 per player,
but this is just a suggestion and ultimately left up to you to decide in your campaign.}

\newplayr{"Cool" is subjective, but this does not excuse stupidity. Never assume the GM will give you Cool Tokens. Roleplay appropriately and see what happens.}

\chapter{\HPful\ and \ENful}

\section{\HPful\index{\HPful}}

\HPful\ is a crucial metric that represents tangable, lasting damage
to a character's body. Too much \HPful\ damage could result in a
\textit{permanent wound.} Your \emph{Maximum \HPful} is equal to your total spent \attrval\footnote{\attrval, x.x.x}.

\newplayr{Keep your guard up! Although most damage can get absorbed by energy, some situations may utilize \emph{Direct Damage}\index{Direct Damage} which can force all of the incoming damage to \HPful !}

\subsection{Permanent Wounds\index{Permanant Wounds}}
\emph{Permanent Wounds} are lasting injuries or conditions that will be with your character throughout their campaign(s).

\subsection{Death\index{Death}}

If a character's \HPful\ falls to or below 0, the character is dead. The good news is that death doesn't have to be permanent!
This may be an opportunity for your friends to go on an adventure
to revive you while you train in the afterlife and learn a new \techn .

However, if the player desires, they may instead create a new character
to be recruited by the adventuring party. Just know that it is a \emph{choice}
and not a neccessity.

\section{\ENful\index{\ENful}}

\ENful\ is a critical resource within \gametitlemini\ with a variety
of uses from executing \techn s  to mitigating critical injury.
Managing how and when to use your \ENful\ is vital to survival. Your \emph{Maximum \ENful} is equal to double your \emph{Maximum \HPful}.

\subsection{\techn\ Resources}\index{\techn}

To use a \techn\footnote{\techn, x.x.x}, an \ENful\ cost must
be paid. The starting cost is 10 \EN\ and with each level of \advancement\footnote{\advancement, x.x.x}\index{\advancement},
the cost increases by 5 \EN.

\subsection{Damage Types and Diversion\index{Diversion}\index{Damage}}
In \gametitlemini\ damage can come in two forms:
\begin{itemize}
	\item Normal Damage \vdots\ Threatening, but mitigatable. Normal damage is able to be shielded by \ENful\ reserves, and can be allocated between \HPful\ and \ENful\ as \emph{the receiver} sees fit.
	\item Direct Damage \vdots\ Terrifying, and very much life-threatening. This type bypasses all defenses and is delivered exclusively to \HPful .\index{Direct Damage}
\end{itemize}

\subsection{Overexertion}

When a character's \ENful\ falls to to 0, the character is overexerting themself. All instances that would normally go to \ENful\ instead go to \HPful . You may choose to pass out at any time when overexerting yourself-- even if there are challenges you are currently participating in.

\subsection{Passing Out\index{Passing Out}}
When you pass out, you may not act in any way until rested or healed.
\newplayr{It is \emph{universally} frowned upon to attack a passed out opponent, and
only the most evil character would spitefully do such an act.}

\chapter{Encounters}
The structure of how events take place are usually unimportant but
are still expected to occur frequently. Hierarchy is occurs as follows:
\begin{enumerate}
	\item {\large Encounter}\index{Encounter} \linebreak\
	This encapsulates the entire situation. \emph{Encounters} proceed over a predetermined or indefinite number of engagements known as \emph{Rounds}. An encounter ends when: 
	\begin{itemize}
		\item The predetermined Round countdown reaches 0
		\item Only one team is able to take Turns
	\end{itemize}
\item {\large Rounds}\index{Round} \linebreak\
	These encapsulate \emph{every} character's Turn. After every character has had their Turn, a new \emph{Round} begins.
\item {\large Turns}\index{Turn} \linebreak\
	These encapsulate the \emph{selected} character's Actions. In most instances each character takes one \emph{Turn} per Round
\item {\large Actions}\index{Action} \linebreak\
	These are how characters actually do things. In most instances each character will take two \emph{Actions} per Turn. Character Actions are grouped as follows:
		\begin{itemize}
			\item Standard \index{Action!Standard} \vdots\ A basic Action (e.g. movement) that can be repeated.
			\item Unique \index{Action!Unique} \vdots\ A difficult Action that cannot be repeated multiple times per Turn.
			\item Double \index{Action!Double} \vdots\ A tiring Action that consumes that character's entire Turn.
		\end{itemize}
\end{enumerate}
